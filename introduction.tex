\chapter{INTRODUCTION} 

\section{Background of the Study}

Nepal, an agricultural country, relies heavily on farming, with more than 67 percent of its population engaged in agricultural activities and of which 62 percent directly rely
upon agriculture as the main occupation. Agriculture functions as the economic foundation that supports the entire nation of Nepal. The agricultural sector supports millions
of people in earning a living and produces substantial National GDP financial output (World Bank, 2016). Many agricultural issues stem from limited resources alongside
insufficient modern farming techniques and improper planning together creating subpar productivity with insufficient yield potentials, reduced productivity and unsatisfactory
yields.

High-quality agricultural products face increasing market demand from both domestic and international customers. The market demands prompt innovative principles together with effective 
resource management systems. The historically valuable farming methods used in rural Nepal fail to produce optimal crop yields at present. Transforming farming practices represents an 
Important chance to integrate innovative solutions that will enable productive growth alongside product quality maintenance.

Smallholder farmers who comprise a major segment of the farming community face restricted access to contemporary agricultural tools together with next-generation technologies. 
Smallholder farmers limited by their financial constraints keep them away from investing in resource-efficient equipment that would boost their output. Soil monitoring methods 
along with advanced irrigation systems are virtually unknown to smallholder farmers despite their clear benefits. Technology developers have not explored enough unexploited capabilities 
of modern agricultural development applications designed to advance farming results.

The successful handling of pests and crop diseases represents essential components of modern farming operations. Farmers face numerous obstacles in their ability to detect developing
threats before they become severe.The inability of farmers to access critical information hinders their ability to make successful choices which causes increased losses from both pests
and plant diseases. By improving monitoring frameworks coupled with enhanced access to real-time data, farmers can successfully achieve improved agricultural resilience

Existing solutions for monitoring Nepalese agricultural fields are inadequate to provide complete crop health evaluations and optimized resource management. The implementation of 
modern technological systems capable of delivering immediate actionable information would revolutionize how farmers implement resource management and generate decisions. Such advanced monitoring systems will create dual benefits by advancing agricultural yields and better managing essential resources including water and fertilizers

Geological changes in Nepal’s agriculture show effects from weather patterns which produce both irregular rainfall and limited seasons. Environmental climatic changes directly
affect soil content's productivity levels and crop sustained quality. Poor environmental conditions demand farmers to adopt sustainable methods capable of adapting to climate
change while maintaining agricultural productivity and quality.

Nepal has begun to integrate artificial intelligence technologies into agricultural practices throughout to address the current issues. Agriculture powered by machine learning,
artificial intelligence alongside remote sensing and predictive analytics provide innovative solutions which enhance both resource management and farming operational effi-
ciency. The use of innovative future-oriented technologies combined with environmentally friendly practices allows Nepal's agricultural industry to develop
into a stronger and better-producing system. Smart farming investments integrated with advocacy for technological adoption will boost agricultural operations while ensuring millions
of rural residents maintain their livelihoods.




\section{Statement of the Problem}

The agricultural sector of Nepal struggles with resource utilization challenges that reduce the efficiency of NPK (Nitrogen, Phosphorus, and 
Potassium) fertilizers alongside water supplies and humidity control measures. Limited access to data on optimal resource allocation results 
in improper  application, which negatively impacts crop growth rates and overall productivity. Soil quality deteriorates alongside reduced 
fertility when fertilizer management remains inadequate and irregular water distribution along with faulty irrigation systems worsens this 
impact. The lack of resource management efficiency creates double harm to agricultural production and consumes farmers' budget while making
it difficult for them to adjust to climate shifts and protect our food supplies.  \\
\\
In addition to resource allocation, Nepalese farmers must deal with crop disease challenges that substantially affect agricultural productivity rates, despite limited funding for resources. Traditional approaches to detecting diseases and controlling their spread produce delayed responses which result in intensified agricultural losses. Current real-time monitoring systems are missing, thus, farmers struggle to identify when blight, rust and mold start spreading between farms. This result in excessive use of pesticides, even causing more harm to the agricultural farm. Biotic crop disease management needs proactive data-oriented technology solutions to enhance both crop sustainability and health outcomes. 

Another critical issue which affects the agricultural productivity is the lack of trust and transparency in decision-making for resourse use, 
both using traditional methods and using AI models for precision farming. The application of Local Interpretable  Model-agnostic Explainations
(Lime) and Shapley Additive Explanations (Shap) comes in hangy for decision makings like fertilizer recommendation and disease prediction. 
Farmer often hesitate to trust the black-box AI model, without having clear insights into how the decisions are made. As a result, they hesitate
to fully trust technological solutions.

\section{Objectives}
% \setlength{\tabcolsep}{0.5in}  % Set tab space to 0.5 inch
The main objective is to design a real-time monitoring system using ESP32 and integrated sensors (rainfall, humidity, NPK, soil moisture)
which collect enviromental and crop data, and leverage AI powered predictive analytics to optimize the allocation of agricultural 
resources (e.g., water, NPK fertilizers)

Secondly,
To design and implement a machine learning model to detect crop diseases based on user input image.

Third,
To apply Explainable AI (XAI) techniques (such as Lime) to improve the interpretability and transparency of disease detection to improve farmers trust.
.
% Introduce your main objective first in one sentence, followed by a bulleted list of specific objectives, left aligned.
% 1.	Follow a 0.5-inch Tab setting.
% 2.	Add more here.

